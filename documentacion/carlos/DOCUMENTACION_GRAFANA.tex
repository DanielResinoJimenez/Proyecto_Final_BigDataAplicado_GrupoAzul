\documentclass{article}
\usepackage[utf8]{inputenc}
\usepackage[T1]{fontenc}
\usepackage[spanish]{babel}
\usepackage{listings}
\usepackage{xcolor}
\usepackage{geometry}
\geometry{a4paper, margin=1in}

\definecolor{codebg}{rgb}{0.95,0.95,0.95}
\lstset{
    backgroundcolor=\color{codebg},
    basicstyle=\ttfamily\small,
    breaklines=true,
    frame=single,
    rulecolor=\color{black}
}

\title{Documentación: Monitoreo de Redis con Prometheus y Grafana en CentOS 7}
\author{}
\date{}

\begin{document}

\maketitle

Este documento describe paso a paso cómo configurar un sistema de monitoreo para Redis usando Prometheus, \texttt{redis\_exporter} y Grafana en un entorno con máquinas virtuales CentOS 7.

\section*{Arquitectura del entorno}
\begin{itemize}
    \item Máquina 1 (\texttt{nodo1}) — IP: \texttt{172.16.200.63} → Servicios: Prometheus, Grafana
    \item Máquina 2 — IP: \texttt{172.16.200.23} → Servicio: Redis
\end{itemize}
Ambas máquinas deben tener conectividad de red entre sí.

\section*{Paso 1: Instalar y configurar Prometheus en nodo1 (172.16.200.63)}

Crear usuario y directorios:
\begin{lstlisting}[language=bash]
sudo useradd --no-create-home --shell /bin/false prometheus
sudo mkdir /etc/prometheus
sudo mkdir /var/lib/prometheus
sudo chown prometheus:prometheus /etc/prometheus
sudo chown prometheus:prometheus /var/lib/prometheus
\end{lstlisting}

Descargar e instalar Prometheus:
\begin{lstlisting}[language=bash]
cd /tmp
LATEST_VERSION=$(curl -s https://api.github.com/repos/prometheus/prometheus/releases/latest | grep '"tag_name":' | cut -d '"' -f 4)
wget https://github.com/prometheus/prometheus/releases/download/$LATEST_VERSION/prometheus-$LATEST_VERSION.linux-amd64.tar.gz
tar xvf prometheus-$LATEST_VERSION.linux-amd64.tar.gz
cd prometheus-$LATEST_VERSION.linux-amd64
sudo cp prometheus /usr/local/bin/
sudo cp promtool /usr/local/bin/
sudo chown prometheus:prometheus /usr/local/bin/prometheus
sudo chown prometheus:prometheus /usr/local/bin/promtool
sudo cp -r consoles/ console_libraries/ /etc/prometheus/
sudo chown -R prometheus:prometheus /etc/prometheus/consoles
sudo chown -R prometheus:prometheus /etc/prometheus/console_libraries
\end{lstlisting}

Configurar archivo de Prometheus:
\begin{lstlisting}[language=yaml]
global:
  scrape_interval: 15s

scrape_configs:
  - job_name: 'prometheus'
    static_configs:
      - targets: ['localhost:9090']

  - job_name: 'redis'
    static_configs:
      - targets: ['172.16.200.23:9121']
\end{lstlisting}

Crear servicio systemd:
\begin{lstlisting}[language=ini]
[Unit]
Description=Prometheus
Wants=network-online.target
After=network-online.target

[Service]
User=prometheus
Group=prometheus
Type=simple
ExecStart=/usr/local/bin/prometheus \
    --config.file /etc/prometheus/prometheus.yml \
    --storage.tsdb.path /var/lib/prometheus/ \
    --web.console.templates=/etc/prometheus/consoles \
    --web.console.libraries=/etc/prometheus/console_libraries

[Install]
WantedBy=multi-user.target
\end{lstlisting}

Iniciar Prometheus:
\begin{lstlisting}[language=bash]
sudo systemctl daemon-reexec
sudo systemctl enable prometheus
sudo systemctl start prometheus
\end{lstlisting}

Verificar en: \texttt{http://172.16.200.63:9090}

\section*{Paso 2: Instalar y configurar redis\_exporter en la máquina de Redis (172.16.200.23)}

Verificar contraseña de Redis:
\begin{lstlisting}[language=bash]
grep "^requirepass" /etc/redis/redis.conf
\end{lstlisting}
(Anota la contraseña; ejemplo: \texttt{password})

Descargar e instalar \texttt{redis\_exporter}:
\begin{lstlisting}[language=bash]
cd /tmp
LATEST=$(curl -s https://api.github.com/repos/oliver006/redis_exporter/releases/latest | grep '"tag_name":' | cut -d '"' -f 4)
wget https://github.com/oliver006/redis_exporter/releases/download/$LATEST/redis_exporter-$LATEST.linux-amd64.tar.gz
tar xvf redis_exporter-$LATEST.linux-amd64.tar.gz
sudo cp redis_exporter-$LATEST.linux-amd64/redis_exporter /usr/local/bin/
sudo useradd --no-create-home --shell /bin/false redis_exporter 2>/dev/null || true
sudo chown redis_exporter:redis_exporter /usr/local/bin/redis_exporter
\end{lstlisting}

Crear servicio systemd:
\begin{lstlisting}[language=ini]
[Unit]
Description=Redis Exporter
After=network.target

[Service]
User=redis_exporter
Group=redis_exporter
Type=simple
ExecStart=/usr/local/bin/redis_exporter \
    --redis.addr=localhost:6379 \
    --redis.password=password \
    --web.listen-address=:9121

Restart=always

[Install]
WantedBy=multi-user.target
\end{lstlisting}

\textbf{Importante}: Reemplaza \texttt{password} con tu contraseña real.

Iniciar el servicio:
\begin{lstlisting}[language=bash]
sudo systemctl daemon-reexec
sudo systemctl enable redis_exporter
sudo systemctl start redis_exporter
\end{lstlisting}

Verificar métricas:
\begin{lstlisting}[language=bash]
curl http://localhost:9121/metrics | grep redis_up
\end{lstlisting}
Debe mostrar: \texttt{redis\_up 1}

\section*{Paso 3: Instalar y configurar Grafana en nodo1 (172.16.200.63)}

Agregar repositorio e instalar:
\begin{lstlisting}[language=bash]
sudo tee /etc/yum.repos.d/grafana.repo <<EOF
[grafana]
name=grafana
baseurl=https://packages.grafana.com/oss/rpm
repo_gpgcheck=1
enabled=1
gpgcheck=1
gpgkey=https://packages.grafana.com/gpg.key
sslverify=1
sslcacert=/etc/pki/tls/certs/ca-bundle.crt
EOF

sudo yum install -y grafana
\end{lstlisting}

Iniciar Grafana:
\begin{lstlisting}[language=bash]
sudo systemctl daemon-reexec
sudo systemctl enable grafana-server
sudo systemctl start grafana-server
\end{lstlisting}

Acceder en: \texttt{http://172.16.200.63:3000} \\
Usuario/contraseña por defecto: \texttt{admin} / \texttt{admin}

\section*{Paso 4: Configurar datasource en Grafana}
\begin{enumerate}
    \item En Grafana, ir a \textbf{Configuration > Data Sources}
    \item Hacer clic en \textbf{Add data source}
    \item Seleccionar \textbf{Prometheus}
    \item En \textbf{URL}, ingresar: \texttt{http://localhost:9090}
    \item Hacer clic en \textbf{Save \& test}
\end{enumerate}

\section*{Paso 5: Importar dashboard de Redis}
\begin{enumerate}
    \item En Grafana, hacer clic en \textbf{+ > Import}
    \item Ingresar el ID: \textbf{763}
    \item Seleccionar el datasource \textbf{prometheus}
    \item Hacer clic en \textbf{Import}
\end{enumerate}

\textbf{Nota}: Si el dashboard muestra "N/A", editar los paneles y asegurarse de que las consultas usen \texttt{\{job="redis"\}} en lugar de \texttt{\{job="redis\_exporter"\}}.

\section*{Verificación final}
\begin{itemize}
    \item \textbf{Prometheus}: \texttt{http://172.16.200.63:9090/targets} → job \texttt{redis} debe estar \textbf{UP}
    \item \textbf{Grafana}: Dashboard de Redis debe mostrar métricas reales:
    \begin{itemize}
        \item Uso de memoria
        \item Clientes conectados
        \item Comandos por segundo
        \item Tiempo de actividad
    \end{itemize}
\end{itemize}

\section*{Solución de problemas comunes}

\textbf{Redis muestra "N/A" en Grafana}
\begin{itemize}
    \item Verificar que \texttt{redis\_exporter} tenga la contraseña correcta
    \item Confirmar que en Prometheus las métricas tengan \texttt{job="redis"}
    \item Editar las consultas del dashboard para usar \texttt{\{job="redis"\}}
\end{itemize}

\textbf{Target de Redis aparece como DOWN en Prometheus}
\begin{itemize}
    \item Verificar firewall en la máquina de Redis: \texttt{sudo firewall-cmd --permanent --add-port=9121/tcp}
    \item Confirmar que \texttt{redis\_exporter} esté escuchando en todas las interfaces (\texttt{:9121}, no \texttt{127.0.0.1:9121})
\end{itemize}

\section*{Notas importantes}
\begin{itemize}
    \item Este documento se centra exclusivamente en el monitoreo de Redis
    \item Todos los pasos asumen una instalación limpia de CentOS 7
    \item Las contraseñas y direcciones IP deben ajustarse según el entorno real
    \item El dashboard ID 763 es el estándar para Redis con Prometheus
\end{itemize}

\end{document}