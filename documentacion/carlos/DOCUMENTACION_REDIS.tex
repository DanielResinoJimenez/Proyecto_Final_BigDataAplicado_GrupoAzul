```latex
\documentclass[11pt]{article}
\usepackage[utf8]{inputenc}
\usepackage[T1]{fontenc}
\usepackage[spanish]{babel}
\usepackage{geometry}
\usepackage{listings}
\usepackage{xcolor}
\usepackage{booktabs}
\usepackage{longtable}
\usepackage{enumitem}
\usepackage{titlesec}
\usepackage{hyperref}
\hypersetup{
    colorlinks=true,
    linkcolor=black,
    filecolor=magenta,      
    urlcolor=cyan,
}

\geometry{a4paper, margin=2.5cm}
\titleformat{\section}{\large\bfseries}{}{0em}{}
\titleformat{\subsection}{\normalsize\bfseries}{}{0em}{}

\definecolor{codegreen}{rgb}{0,0.6,0}
\definecolor{codegray}{rgb}{0.5,0.5,0.5}
\definecolor{codepurple}{rgb}{0.58,0,0.82}
\definecolor{backcolour}{rgb}{0.95,0.95,0.92}

\lstdefinestyle{mystyle}{
    backgroundcolor=\color{backcolour},   
    commentstyle=\color{codegreen},
    keywordstyle=\color{magenta},
    numberstyle=\tiny\color{codegray},
    stringstyle=\color{codepurple},
    basicstyle=\ttfamily\small,
    breakatwhitespace=false,         
    breaklines=true,                 
    captionpos=b,                    
    keepspaces=true,                 
    numbers=left,                    
    numbersep=5pt,                  
    showspaces=false,                
    showstringspaces=false,
    showtabs=false,                  
    tabsize=2
}

\lstset{style=mystyle}

\title{Guía Paso a Paso: Configuración y Uso Básico de Redis para Almacenamiento de Datos de Sensores}
\author{}
\date{}

\begin{document}

\maketitle

Esta guía documenta el proceso completo para configurar, autenticar, verificar y gestionar datos en Redis, enfocándose en un escenario típico de almacenamiento de lecturas de sensores. Incluye solo los pasos esenciales y decisiones relevantes.

\section*{1. Conexión y Autenticación}

Redis está configurado con una contraseña (directiva \texttt{requirepass} en \texttt{redis.conf}). Para interactuar con él, debes autenticarte.

\subsection*{Opción recomendada (segura y simple):}
\begin{lstlisting}[language=bash]
redis-cli
127.0.0.1:6379> AUTH tu_contraseña
OK
\end{lstlisting}

\subsection*{Alternativa para comandos individuales (con advertencia de seguridad):}
\begin{lstlisting}[language=bash]
redis-cli -a "tu_contraseña" PING
\end{lstlisting}
\textbf{Advertencia:} Usar \texttt{-a} expone la contraseña en el historial del shell. Solo para entornos de desarrollo.

\subsection*{Forma más segura para scripts:}
\begin{lstlisting}[language=bash]
export REDISCLI_AUTH="tu_contraseña"
redis-cli PING
\end{lstlisting}

\section*{2. Verificar Contenido Actual de la Base de Datos}

\subsection*{Listar todas las claves existentes:}
\begin{lstlisting}[language=bash]
redis-cli -a "tu_contraseña" KEYS "*"
\end{lstlisting}
Esto muestra todos los nombres de claves (ej. \texttt{sensor\_id\_1}, \texttt{sensor\_id\_2}, ...). \\
\textbf{Advertencia:} No usar en producción con millones de claves; bloquea Redis temporalmente.

\subsection*{Ver el valor de una clave específica (tipo string):}
\begin{lstlisting}[language=bash]
redis-cli -a "tu_contraseña" GET sensor_id_19
# Ejemplo de salida: "Exito_Total"
\end{lstlisting}

\subsection*{Ver el tipo de una clave:}
\begin{lstlisting}[language=bash]
redis-cli -a "tu_contraseña" TYPE sensor_id_19
# Salida esperada: string
\end{lstlisting}

\section*{3. Listar Todo el Contenido (Claves + Valores)}

Si todas tus claves son strings (como en el caso de sensores), usa este script en Bash:

\begin{lstlisting}[language=bash]
#!/bin/bash
PASSWORD="tu_contraseña"

for key in $(redis-cli -a "$PASSWORD" KEYS "*"); do
  value=$(redis-cli -a "$PASSWORD" GET "$key")
  echo "$key = $value"
done
\end{lstlisting}

Guarda como \texttt{list\_redis.sh}, hazlo ejecutable (\texttt{chmod +x list\_redis.sh}) y ejecuta.

\section*{4. Configuración de Persistencia y Manejo de Errores}

\subsection*{Desactivar bloqueo por errores de guardado en disco}
Por defecto, Redis detiene las escrituras si falla la persistencia (\texttt{BGSAVE}). Para evitar esto (útil en desarrollo):

\begin{lstlisting}[language=bash]
redis-cli -a "tu_contraseña" CONFIG SET stop-writes-on-bgsave-error no
\end{lstlisting}

\textbf{Cuándo usarlo:} Entornos de desarrollo, caché no crítico, o cuando tienes respaldo externo. \\
\textbf{No usar en producción} si los datos son críticos y dependes de la persistencia de Redis.

\subsection*{Verificar configuración actual:}
\begin{lstlisting}[language=bash]
redis-cli -a "tu_contraseña" CONFIG GET stop-writes-on-bgsave-error
\end{lstlisting}

\section*{5. Gestión de Memoria (Evitar Pérdida de Datos)}

Redis puede eliminar claves automáticamente si se alcanza el límite de memoria.

\subsection*{Verificar límite de memoria y política de evicción:}
\begin{lstlisting}[language=bash]
redis-cli -a "tu_contraseña" INFO memory | grep maxmemory
redis-cli -a "tu_contraseña" CONFIG GET maxmemory-policy
\end{lstlisting}

\subsection*{Recomendación para datos de sensores:}
\begin{itemize}[left=0pt]
    \item Si los datos deben conservarse, asegúrate de que \texttt{maxmemory} sea suficiente.
    \item Usa la política \texttt{noeviction} para evitar eliminación automática (pero monitorea el uso de RAM):
\end{itemize}

\begin{lstlisting}[language=bash]
redis-cli -a "tu_contraseña" CONFIG SET maxmemory-policy noeviction
\end{lstlisting}

\section*{6. Diagnóstico de Datos Faltantes}

Si esperas datos que no aparecen en Redis:

\begin{enumerate}[left=0pt]
    \item \textbf{Verifica que la aplicación envía los comandos correctamente} (logs del cliente).
    \item \textbf{Revisa los logs del servidor Redis:}
    \begin{lstlisting}[language=bash]
sudo tail -f /var/log/redis/redis-server.log
    \end{lstlisting}
    \item \textbf{Confirma que no se está superando \texttt{maxmemory}} (paso 5).
    \item \textbf{Asegúrate de que la persistencia no está causando confusión} (los datos están en memoria aunque no se hayan guardado en disco aún).
\end{enumerate}

\textbf{Importante:} Redis \textbf{no almacena ni registra} intentos fallidos de escritura (ej. por autenticación incorrecta). Solo guarda lo que se escribe exitosamente.

\section*{Resumen de Comandos Útiles}

\begin{center}
\begin{tabular}{ll}
\toprule
\textbf{Acción} & \textbf{Comando} \\
\midrule
Autenticar & \texttt{AUTH tu\_contraseña} \\
Listar claves & \texttt{KEYS "*"} \\
Obtener valor & \texttt{GET nombre\_clave} \\
Ver tipo & \texttt{TYPE nombre\_clave} \\
Desactivar bloqueo por error de disco & \texttt{CONFIG SET stop-writes-on-bgsave-error no} \\
Ver uso de memoria & \texttt{INFO memory} \\
Cambiar política de evicción & \texttt{CONFIG SET maxmemory-policy noeviction} \\
\bottomrule
\end{tabular}
\end{center}

\end{document}
```